Persian is a member of the Indo-European language family, and uses Arabic letters for writing. %The language has a rich morphology system and also many complexities in orthography. 
Seraji et al.~\shortcite{seraji2012} provide a comprehensive overview on the main characteristics of the language. %For example, they show how the word ``the libraries'' can be written in up to 12 different writing styles. 
For instance, the diacritic signs are not written---it is expected that the reader can read the text in the absence of the short vowels. This characteristic causes a special kind of word ambiguity in writing, such that some words are pronounced differently while their written forms are the same e.g., \<k^sty> ``wrestling'' and ``ship''.

%\paragraph{Orthography} Among other languages, Persian is a language with challenging preprocessing tasks. Persian writing is based on Arabic letters plus foura additional ones:. In Persian characters have different forms depending on their position in the word. Based on how they connect to eacht other, there are two categories \cite{seraji2012}: 1) dual-joining: Characters have two different shapes based on their position in the sentence like sin  2) right-joining: These characters do not accept any character from left-hand side like r, z in norouz. Further, in Persian languages we have semi-spaces (zero-width non-joiner spaces). In case of wrong usage, this causes disambiguity. For instance "daarabaad" meaning a known place in Tehran, which with regular spaces means a built place. This causes errors in correct processing of NLP tools.

%\paragraph{Morphological difference} There is no grammatical gender for suffixing affixational morphology of Persian language \cite{amtrup2000}. With adjectives, a suffix is added to the adjective indicating its comparative (adding tar) or superlative forms (adding tarin). For instance for adjective sard (cold), we have sard-tar and sardtarn. In nouns, the plural form is made by adding ha or an to the noun like derakht (tree) and derakhtha or derakhtan. Comparable to 'a/an' in English, Persian has  (ie) which give the meaning of one: derakhti (one three). In verbs, are used to convey tenses, mood and subject. Usually the past tense has the same stem of the verb, and by adding another verb the future form is made, e.g, mikhoram, khordam, khaham khord. Negation is made by adding n (not) at the beginning of the verb, keeping the verb as one word. However, when it comes to compound verbs, the constructions are mostly irregular (combination of nouns and adjectives). 

%Saedi et al.\cite{saedi2009} propose an automatic Persian/English translator containing two modules for En to Pr and Pr to En translation. To perform WSD in the first module, they extend the Lesk algorithm~\cite{lesk1986automatic} by assigning different scores to the words in the gloss. In Lesk algorithm, neighborhood  of a word is assumed to share a common topic with the word. In scoring they consider POS ad sense tags of each word. They utilize combination of rule-based and semantic methods to improve the result.  An important catalog to extend the word structures is WordNet, which groups English words with structured sets. 
 
In recent years, several tools and libraries were introduced, targeting the complexities of Persian language in NLP. For example, Seraji et al.~\shortcite{seraji2012} provides a set of tools for preprocessing (Pre-Per), sentence segmentation and tokenization (SeTPer), and also POS tagging (TagPer). Dehdari et al.~\shortcite{dehdari2008link} brings forward a stemmer and morphological analyzer, called PerStem. More recently, Samvelian et al.~\shortcite{samvelian2014extending} introduces PersPred, focusing on processing of compounding verbs. Finally, Feely et al.~\shortcite{feely2014cmu} provides a front-end and new tools for language processing. In this work, similar to Jadidinejad et al.~\shortcite{jadidinejad2010evaluation}, we use PerStem for stemming, together with TagPer as a state-of-the-art POS tagging tool.

In addition to the NLP tools, knowledge and data resources are an important part of WSD and CL-WSD solutions. The main knowledge resource in Persian is \emph{FarsNet}~\cite{shamsfard2010semi}---the Persian WordNet. Its structure is comparable to WordNet and goes by the same principles while containing significantly fewer words ($\sim$13K versus $\sim$147K). Also, most of its synsets are mapped to synsets in WordNet using equal or near-equal relations.%, resulting in connection between Persian and English synsets. 

While the knowledge-based systems are limited %to the borders of the knowledge resource and not 
and only at high cost extendable to more specific domains, exploiting  parallel corpora can be another effective method for CL-WSD. The existing parallel corpora (English-Persian) are as follows: Tehran English-Persian Parallel (TEP) ~\cite{pilevar2011tep} corpus---a free collection extracted from 1600 movie subtitles. The Parallel English-Persian News (PEN)~\cite{farajian2011pen} corpus aligns 30K sentences of news corpora. However, to the extent of our knowledge, this collections is not yet available. Finally, the collection provided by European Language Resource Association (ELRA) which is a commercial collection with approximately 100K aligned sentences. Among the mentioned resources, TEP is the only publicly available one, but it only contains informal conversations, and therefore it does not provide a general representation of the language.% for NLP solutions. 

In the absence of reliable and comprehensive resources, our unsupervised CL-WSD method exploits the use of monolingual corpora. The available text collections in Persian are as follows: The \emph{Hamshahri} collection~\cite{aleahmad2009hamshahri}, a widely used Persian collection, containing approximately 318K news articles of the Hamshahri newspaper. The articles are of various subjects of economy, sport, politics, psychology, literature, or art from 1996 to 2007. The collection was introduced as the main resource for the Persian task of the CLEF Ad Hoc track~\cite{ferro2010clef,agirre2009clef} in 2008/2009. The \emph{dotIR} collection\footnote{\url{http://ece.ut.ac.ir/DBRG/webir/index.html}}, released in 2008, is created by crawling ~1000K web pages in the .ir domain. Finally,  \emph{Bigjekhan}\footnote{\url{http://ece.ut.ac.ir/dbrg/Bijankhan}} and Uppsala Persian Corpus (UPEC)~\cite{seraji2012} are smaller collections with manually tagged POS data. 

Between the Hamshahri and dotIR (as bigger collections), since the Hamshahri collection is more recent and  has also been used more in the community of Information Retrieval (IR) and NLP, we selected it as the main resource for our experiments. In addition, in comparison to the dotIR collection, as the content of the Hamshahri collection contains revised newspaper articles, we assume that it is a better representation of the language. 

%The task is based on Hamshahri collection~\cite{aleahmad2009hamshahri} with size of 700 MB and 160,000 documents with various topics.

In terms of related work addressing the CL-WSD problem in Persian, Sarrafzadeh et al.~\cite{sarrafzadeh2011} follows a knowledge-based approach by exploiting FarsNet together with leveraging English sense disambiguation. %Their model consists of three phases of: English sense disambiguation, utilizing WordNet and FarsNet to transfer the sense, and selecting the sense from FarsNet. As another method, they investigate direct WSD by applying extended Lesk algorithm~\cite{lesk1986automatic} for Persian WSD. They count the number of shared words between two glosses, the gloss of each sense of the target word with the gloss of other words in the phrase. The one with larger number of common words is chosen. 
%They test on parallel page of Wikipedia in English and Persian evaluated by experts. %Finally, they show that the first approach works better since they can use the state of the art disambiguater of English language and direct approach suffers from lack of NLP tools and ambiguity of Farsi words. 
However, since they evaluate their methods only internally, the results are impossible to compare with other possible approaches. %This lack of standard and reproducible evaluation in Persian, especially in the NLP and MT domain can be seen in other work~\cite{motazedi2009english}. 
%mihai:perhaps add back this last sentence if space allows, but it doesn't say much.

In this work, we address this shortage by creating a new CL-WSD benchmark  for Persian, based on the SemEval 2013 CL-WSD task. We then report the result of our unsupervised approach on the provided benchmark and compare it with a state-of-the-art CL-WSD system.